%*******************************************************************************
%*********************************** First Chapter *****************************
%*******************************************************************************

\chapter{Numerical Optimization: Introduction}
\label{chapter:optimization}

\nomenclature[x-R]{$\mathbb{R}$}{Real space}
\nomenclature[x-S]{$\mathit{S}$}{Variable space}
\nomenclature[x-M]{$\mathcal{M}$}{A Manifold}
\nomenclature[x-T]{$T_x\mathcal{M}$}{The tangent space of manifold $\mathcal{M}$ at point x}
\nomenclature[a-F]{$F$}{The set of linearized feasible directions}
\nomenclature[a-f]{$f$}{Objective function}
\nomenclature[a-c]{$c_i$}{Constraint function}
\nomenclature[a-x]{$x$}{Optimization variable}
\nomenclature[a-xstar]{$x^*$}{Solution of the optimization problem}
\nomenclature[a-E]{${E}$}{Set of index for which constraints are equality constraints}
\nomenclature[a-I]{${I}$}{Set of index for which constraints are inequality constraints}
\nomenclature[x-L]{$\mathcal{L}(x,\lambda)$}{Lagrangian function of the optimization problem}
\nomenclature[G-o]{$\Omega$}{Feasible set}
\nomenclature[z-st]{s.t.}{subject to}
\nomenclature[z-KKT]{KKT}{Karush-Kuhn-Tucker first order optimality conditions}
\nomenclature[z-LICQ]{LICQ}{Linear Independence Constraints Qualification}
\nomenclature[z-QP]{QP}{Quadratic Programming}
\nomenclature[z-IQP]{IQP}{Inequality constrained Quadratic Programming}

\ifpdf
    \graphicspath{{Chapter1/Figs/Raster/}{Chapter1/Figs/PDF/}{Chapter1/Figs/}}
\else
    \graphicspath{{Chapter1/Figs/Vector/}{Chapter1/Figs/}}
\fi

\section{List of contributions}
\begin{itemize}
  \item \sout{Introduction to optimization}
  \item \sout{Notions of unconstrained optimization}
  \item \sout{Notions of Constrained Optimization}
  \item \sout{KKT}
  \item Line Search
  \item Trust Region
  \item Filter
  \item Merit function
  \item SQP
  \item Restoration phase
  \item \sout{Hessian Approximation: BFGS, SR1}
\end{itemize}

%********************************** First Section *********************************
\section{Introduction}
In modern science, optimization has a very important place.
Mechanical engineers optimize the shape of stuctural parts.
Investors optimize the profit of a portfolio while minimizing the risks of loss.
Chemists optimize the efficiency and speed of reactions.
When it comes to robotics, optimization is everywhere.
From the design of a robot to its actuation.
Any positioning of a robot requires the computation of the articular parameters of each joint of the robot, finding such parameters might be possible by using analytical methods for very simple robots, but for robots as complex as humanoid robots, it is not possible.
Most often, an optimization process is used.

The goal of an optimization algorithm is to find an optimal solution $x^*$ to a problem.
Optimal in the sense that the solution is an optimum of a given objective function $f$.
And solution of a problem in the sense that is satisfies a set of $m$ constraints $\{c_i,\ i\in [1,m]\}$.
Both the constraints and the objective function are defined on the variable space $\mathit{S}$, which is the space in which the variable $x$ lives and in which we search a solution $x^*$ to our problem.

In order to present the principles of optimization, in this chapter we will always consider that the variable space is $\mathit{S}=\mathbb{R}^n$.
We first consider unconstrained problems, that do not have any constraint and that only goal is to minimize a cost function.
That type of problem will not be used in the rest of our dissertation, therefore, we will just present them shortly.
We will then focus on constrained optimization problems and the numerous methods used in their resolution.
We will particularly detail one specific constrained problem resolution algorithm that is the Sequential Quadratic Program(SQP).
The extension of those methods to solving problems on more complex variable spaces that are the non Euclidean manifolds will be the topic of a further chapter of this dissertation.

\section{Unconstrained Optimization}

An unconstrained optimization problem consists in minimizing an objective function without any constraint.
This problem that we denote $\mathcal{P}$ can be formulated as follows:

\begin{equation}
  \min_{x\in\mathbb{R}^n}\ {f(x)}
\label{eq:unconstrainedOptim}
\end{equation}

In order to solve this problem, we want to design an iterative algorithm that, starting from an initial guess $x_0$, will converge toward the solution $x^*$.
The objective function is not necessarily completely known.
In the sense that we cannot always have an explicit formula, often, the function $f$ is computed by another program that is able to compute $f(x)$, $\nabla_x f(x)$ and sometimes $\nabla_{xx} f(x)$ for a given value of $x$.
In order to have an efficient algorithm, we want to avoid any unnecessary computation of $f(x)$ and its derivatives.
We will denote the values taken by $x$ along the iterations as $x_0$, $x_1$, $x_2$,\ldots $x_i$.
And $f(x_i)$ is denoted $f_i$.

Since our knowledge of the objective function is only partial, it would not be possible to guarantee that a point $x^*$ is a global solution:

\begin{equation}
  \text{$x^*$ is a global solution of $\mathcal{P}$ on $\mathit{S}$ } \equiv \forall x \in \mathit{S}, f(x^*) \leq f(x)
\end{equation}

Though, we can find a local solution $x^*$ of $\mathcal{P}$ (a local minimizer of $f$) such that:

\begin{equation}
  \text{There exist a neighborhood } \mathit{N}\text{ of }x^*\text{ such that
  }\forall x\in \mathit{N}, \ f(x^*) \leq f(x)
\end{equation}

That is the kind of solution that we are looking for and that our algorithms should find.

Under the assumption that the objective function is smooth and sufficiently continuous ($\mathcal{C}_2$), then we have the following sufficient conditions for the optimality of $x^*$ as presented in \cite{nocedal:book:2006}:

\begin{theorem}
  If $\nabla^2f$ is continuous in an open-neighborhood of $x^*$ and that $\nabla f(x^*)=0$ and $\nabla^2 f(x^*)$ is positive definite.
  Then $x^*$ is a strict local minimizer of $f$.
  \label{optimalityTheorem}
\end{theorem}

%\subsection{Globalization methods}

%During the resolution of an optimization problem, the algorithm will generate a
%sequence of iterates $x_k$ starting from the initial iterate $x_0$ (Which is
%usually provided by the user). During the step $k$ of the optimization process,
%the solver, the current point is $x_k$ and we try to find $x_{k+1}$ such that
%$f(x_{k+1}) < f(x_k)$. There are several strategies to do that but we will only focus
%on 2 of them that are the most popular, the line-search and trust-region
%methods.

%\subsubsection{The Line-Search Strategy}
%In the Line-Search strategy, given a point $x_k$, a descent direction $p_k$ from this
%point is chosen and then a length of step is calculated to minimize the
%following 1-dimentional problem:
%\begin{align}
  %\min_{\alpha \in \mathbb{R}^+} f(x_k + \alpha.p_k)
%\label{eq:lineSearchNLP}
%\end{align}

%Once the best value of $\alpha$ has been found, the next iterate is computed:
%$x_{k+1} = x_{k} + \alpha p_k$ and the same process is repeated until a
%satisfying solution is found.

%It is not always necessary to find the optimal value of $\alpha$, especially if
%that is expensive. Indeed, $\alpha$ is only used to calculate the next iterate,
%from which another $p_k$ and $\alpha$ will be calculated. And in the end, the
%imprecision on the computation of $\alpha$ will be errased in the other steps of
%the resolution.

%There are several ways to choose a descent direction from an iterate $x_k$. The
%most obvious one is probably the steepest descent direction $-\nabla f_k$. This
%method provides the direction along which f decreases most rapidly and only
%requires the evaluation of the first derivative of $f$, but that method can
%become extremely slow on complicated problems. Another popular approach is the
%Newton method, in which the objective function is approximated to the second
%order

%\begin{equation}
  %f(x_k+p) = f_k + p^T\nabla f_k + p^T\nabla^2f_k p
%\end{equation}

%Then the chosen descent direction is the optimum of that approximated function,
%the Newton direction.
%\begin{equation}
  %p^N_k = -(\nabla^2 f_k)^{-1} \nabla f_k
%\end{equation}
%The choice of this descent direction implies that the $\nabla^2f_k$ is positive
%definite, in which case an adaptation of the definition of $p_k$ is required. Or
%an approximation $B_k$ of $\nabla^2f_k$ that guaranties definite positiveness can be
%used. For example, the symmetric-rank-one(SR1) formula and the BFGS(Broyden,
%Fletcher, Goldfarb, Shanno) formula. Then the step becomes

%\begin{equation}
  %p_k = -B_k^{-1}\nabla f_k
%\end{equation}

%\subsubsection{The Trust-Region Strategy}
%The Trust-Region Strategy works in an opposite way than the line-search one in
%the sense that during a line-search step, a direction is chosen, and based on
%that direction, a step-length is chosen. Whereas with a Trust-Region approach, a
%maximum step-length is chosen, and based on it, the descent direction and lenght
%are chosen.
%The principle of the trust region is that along the optimization process, a
%model of the problem is constructed and enriched at every step and at each step,
%the next iterate is the optimum of the model, with the constraint that the step
%to get there lies inside the trust-region. For example, let us considere that
%the trust region is a sphere of center $x_k$ and radius $\rho_k$, then the
%constraint on $p_k$ is $\|p_k\| \leq \rho_k$. A usual model to take for the
%objective function is the quadratic model with approximated Hessian
%\begin{equation}
  %m_k(p) = f(x_k+p) = f_k + p^T\nabla f_k + p^TB_k p
%\end{equation}
%And the optimization problem to solve at each step of the optimization is

%\begin{align}
  %\min_{p} & \quad f_k + p^T\nabla f_k + p^TB_k p \nonumber\\
%\text{s.t.}&
%\quad \|p\| \leq \rho_k
%\label{eq:trustRegionNLP}
%\end{align}

%Once the solution $p_k$ to this quadratic problem is found, its quality is
%estimated by evaluating the value of $f(x_k+p_k)$. If the actual decrease of $f$
%is satisfying (compared to the decrease predicted by the model) then the step is
%accepted and the size of the trust-region can be increased. Otherwise, the step
%is refused, the trust-region radius is reduced and a new step from $x_k$ is
%computed on that smaller trust region.

\section{Constrained Optimization}

Solving a constrained optimization problem consists in minimizing a cost function while satisfying a set of constraints.

A general formulation for such problems (that we denote $\mathcal{P}$) is:

\begin{equation}
  \label{formulation_NLCP}
  \mathcal{P} \equiv
  \left\{
  \begin{array}{l}
    \min\limits_{x\in\mathbb{R}^n}{f(x)}\\
    \text{ s.t. }
    \left\{
    \begin{array}{l}
      c_i(x) = 0,\ \forall i\in{E}\\
      c_i(x) \geq 0,\ \forall i\in{I}\\
    \end{array}
    \right.
  \end{array}
  \right.
\end{equation}

Where $f$ and $c_i$ are real valued functions on a subset of $\mathbb{R}^n$.
$f$ is the objective function. And $c_i$ are the constraint functions.
${I}$ and ${E}$ are sets of index such that $c_i,\ i\in{E}$ are the equality constraints, and $c_i,\ i\in{I}$ are the inequality constraints.

We define the feasible set $\Omega$ that contains all the feasible points (points satisfying the constraints) of $\mathcal{P}$ \ref{formulation_NLCP}.

\begin{equation}
  \Omega = \left\{ x\in \mathbb{R}^n:\ \forall i\in {E},\ c_i=0,\ \forall i\in{I},\ c_i \geq0\right\}
\end{equation}

The formulation \ref{formulation_NLCP} can then be rewritten as:
\begin{equation}
  \label{formulation_NLCP_compact}
  \mathcal{P} \equiv \min_{x\in\Omega}{f(x)}
\end{equation}

In the general case, the problem \ref{formulation_NLCP_compact} has multiple local solutions.
And finding the global minimizer of $f$ in $\Omega$ is a difficult problem that we do not treat in this thesis.
Our goal is to find a local minimizer $x^*$ of $f$ in $\Omega$.

\subsection{Optimality conditions}

\paragraph{First-Order Optimality Conditions}

The First Order Optimality Condition (Karush-Kuhn-Tucker Condition) is a necessary condition verified by all solutions of problem \ref{formulation_NLCP}.

We introduce the Lagrangian function of \ref{formulation_NLCP}:
\begin{equation}
  \mathcal{L}(x,\lambda) = f(x) + \sum_{i\in E\cup I}\lambda_i c_i(x)
\end{equation}

Any given constraint $c_i$ is said to be active at $x$ if $c_i(x)=0$.
In particular, for any feasible point $x$, all equality constraints $c_i,\ i\in E$ are active. 
An inequality constraint $c_i,\ i\in I$ is inactive if $c_i(x)>0$ and active if $c_i(x) = 0$

\begin{definition}  
  \label{active_set}
  The active set $\mathit{A}(x)$ at a feasible point $x$ is the set of all the indexes of active constraint.
  \begin{equation}
    \mathit{A}(x)=E\cup\{i\in I: c_i(x) = 0\}
  \end{equation}
\end{definition}

Most optimization algorithms make the assumption that at the solution, the constraints satisfy the Linear Independence Constraints Qualification (LICQ)

\begin{definition}
  Given a point $x$ and an active set $\mathit{A}(x)$, the LICQ holds if the set of active constraint gradient $\{\nabla c_i(x),\ i\in \mathit{A}(x)\}$ is linearly independent
\end{definition}

In particular, if any gradient of a constraint is null, then the LICQ does not hold. 
The LICQ should be taken into account during the formulation of a problem.

\begin{theorem}(First-Order Necessary Conditions)\\
  \label{KKT_conditions}
  Suppose that $x^*$ is a local solution of \ref{formulation_NLCP}, that the functions $f$ and $c_i$ are continuously differentiable, and that the LICQ holds at $x^*$.
  Then there is a Lagrange multiplier vector $\lambda^*$, with components $\lambda_i^*$, $i\in E\cup I$, such that the following conditions are satisfied at $(x^*,\lambda^*)$
  \begin{equation}
  \begin{array}{ll}
    \nabla_x\mathcal{L}(x^*,\lambda^*) = 0 &, \\
    c_i(x^*) = 0 &,\ \forall i\in E\\
    c_i(x^*) \geq 0 &,\ \forall i\in I\\
    \lambda_i^* \geq 0 &,\ \forall i\in I\\
    \lambda_i^* c_i(x^*)=0 &,\ \forall i \in E\cup I\\
  \end{array}
  \end{equation}
\end{theorem}

In many cases, the main goal of an optimization algorithm is to find a point that satisfies the KKT conditions.

\paragraph{Second-Order Optimality Conditions}

\begin{definition}
  Given a feasible point $x$, and the active set $\mathit{A}(x)$, the  set of linearized feasible directions $F(x)$ is:
  \begin{equation}
    F(x)=\left\{d\left|
        \begin{array}{ll}
          d^T\nabla c_i(x) = 0&,\ \forall i\in E \\
          d^T\nabla c_i(x) \geq 0&,\ \forall i\in \mathit{A}(x)\cap I \\
        \end{array}
        \right.
    \right\}
  \end{equation}
\end{definition}

And at a KKT solution $(x^*,\lambda^*)$ the critical cone $C(x^*, \lambda^*)$ is:

\begin{equation}
  C(x^*,\lambda^*) = \{w\in F(x^*)|\nabla c_i(x^*)^Tw=0, \forall i\in\mathit{A}(x^*)\cap I \text{ with } \lambda_i^*>0\}
\end{equation}

The critical cone contains the linearly feasible directions for which it is not clear from first derivative information alone if $f$ will increase or decrease.

Once the KKT condition is reached for a point $(x^*, \lambda^*)$.
For a direction $w$ of $F(x^*)$ for which $w^T\nabla f(x^*)=0$, it is impossible to tell from first derivative information only, if an increment in this direction will have a positive effect on the problem resolution.
Thus it is possible that the KKT constaint is not enough to determine if a point is a solution or just a stationary point.

The Second-Order Sufficient Conditions allow to discriminate local solutions from stationary points:

\begin{theorem}
  Suppose that for some feasible point $x^*\in \mathbb{R}^n$ there is a Lagrange multiplier vector $\lambda^*$ such that the KKT conditions are satisfied. Suppose also that
  \begin{equation}
    w^T\nabla_{xx}^2\mathcal{L}(x^*,\lambda^*)w>0,\ \forall w\in C(x^*,\lambda^*),\ w\neq 0
  \end{equation}
  Then $x^*$ is a strict local solution for \ref{formulation_NLCP}
\end{theorem}

In particular, this condition is always verified if $\nabla_{xx}^2\mathcal{L}(x^*,\lambda^*)$ is positive definite.

\section{Resolution of a Non Linear Constrained Optimization Problem}

In the previous section, we presented some theoretical tools to describe the solution of an NLCP that only use the derivative terms of order one and two of the problem's functions.
Here we present some methods for acually solving such problems.
There exists many different algorithms to solve an NLCP.
But all have in common to be iterative processes.
In which we start from an initial guess $x_k$ for $x^*$.
From the first and (sometimes) second-order informations on the functions $f$ and $c_i$ that constitute the problem \ref{formulation_NLCP}, we find a suitable increment $z$ such tha $x_k+z$ is closer to the solution than $x_k$.
And we iterate that operation until a point $x_k$ sufficiently close to a solution is found.
By sufficiently close to a solution we mean that this point satisfies the optimality conditions with a good enough precision.

The resolutions algorithms differ mostly by the method chosen to generate a satisfactory increment $z$. We will list here some of the most popular methods:

\paragraph {The penalty method} combines the cost and constraints function in a penalty function that, as its name indicates, penalizes the violation of constraints, without completely proscribing it.
The penalty function can be written as follows, using the notation $[y]^- = \max\{0, -y\}$:

\begin{equation}
  \label{penalty_function}
  p(\mu, x) = f(x) + \mu \sum_{i\in E}|c_i(x)| + \mu \sum_{i\in I} [c_i(x)]^-
\end{equation}

With this approach, an unconstrained optimization approach can be used.
The minimum of $p(\mu,x)$ varies with the penalty parameter $\mu$.
By increasing $\mu$ to $\infty$, we penalize the violation of the constraints with increasing severity until reaching a solution $x^*$.

\paragraph{The interior point method} generates steps by solving a relaxed constrained problem where slack variables are introduced to relax inequality constraints.
The problem to solve only contains equality, boundary constraints and a cost function a cost function that prevents the slack variables from getting too close to 0:

\begin{equation}
  \min_{x,s}{ f(x)- \mu\sum_{i\in I} \log(s_i)}
  \text{ subject to }
  \left\{
    \begin{array}{l}
     c_i = 0,\ i\in E\\
     c_i - s_i = 0,\ i\in I\\
     s_i \geq 0,\ i\in I
  \end{array}
  \right.
\end{equation}

\paragraph{The Sequential Quadratic Programming (SQP)} is a method in which, at each iterate $(x_k, \lambda_k)$, the increment $z$ is found by solving a Quadratic Programming (QP).
The QP is an approximation of the KKT conditions \ref{KKT_conditions} of the actual problem \ref{formulation_NLCP} to the first order:

\begin{equation}
  \label{approx_QP}
  \begin{array}{ll}
    \min\limits_{z\in \mathbb{R}^n}{} & \frac{1}{2}z^T\nabla_{xx}^2\mathcal{L}(x_k, \lambda_k)z + \nabla f(x_k)^Tz \\
    \text{subject to } & \nabla c_i(x_k)^Tz+c_i(x_k)=0,\ i\in E \\
                       & \nabla c_i(x_k)^Tz+c_i(x_k)\geq 0,\ i\in I
  \end{array}
\end{equation}

Given $z^*$ the solution of \ref{approx_QP}, it can be tempting to directly take the next iterate as $x_{k+1} = x_k + z^*$.
But using that method as is can be problematic as the length of the iterate needs not be bounded.
Thus it is possible to generate a very large step that satisfies the QP.
The QP only approximates the original problem locally.
So taking a too big step can get us further from the solution than we previously were.
To paliate to that issue, mainly two methods are used:
\begin{itemize}
  \item The line-search method: The solution $z$ of \ref{approx_QP} is viewed as a direction and we search a parameter $\alpha\in [0;1]$ so that the next iterate $x_k + \alpha z$ is optimal for the original problem.
  \item The trust-region method adds a set of bound constraints to the QP \ref{approx_QP}. So that the length of the step is limited by the trust-region. The size $\rho$ of the trust region is modified along the iterations based on the estimated quality of the QP approximation. In that case, the QP becomes:
\begin{equation}
  \label{approx_QP_TR}
  \begin{array}{ll}
    \min\limits_{z\in \mathbb{R}^n}{} & \frac{1}{2}z^T\nabla_{xx}^2\mathcal{L}(x_k, \lambda_k)z + \nabla f(x_k)^Tz \\
    \text{subject to } & \nabla c_i(x_k)^Tz+c_i(x_k)=0,\ i\in E \\
                       & \nabla c_i(x_k)^Tz+c_i(x_k)\geq 0,\ i\in I\\
                       & |z|<\rho
  \end{array}
\end{equation}
\end{itemize}

\section{Sequential Quadratic Programming}
\label{sec:sequential_quadratic_programming}

The Sequential Quadratic Programming method is one of the most effective method to solve small and large scale non-linear constrained optimization problems.
It has the upper hand on other methods mostly when solving problem with highly non-linear constraints.

\subsection{Principle}
\label{sub:principle}

As we stated before, the main idea behind the SQP approach it to solve a subproblem that approximate linearly to the first order the KKT conditions.
For simplicity, we considere a problem without inequality constraints.
We denote $C$ the vector of equality constraints.
We denote $z_x$ and $z_\lambda$ the increments on $x$ and $\lambda$ respectively.
\begin{equation}
  \begin{array}{l}
    x_{k+1} = x_k + z_x\\
    \lambda_{k+1} = \lambda_k+z_\lambda\\
  \end{array}
\end{equation}

The problem is the following:
\begin{equation}
  \begin{array}{l}
    \min\limits_{x\in\mathbb{R}^n}{f(x)} \\
    \text{ subject to } C(x) = 0
  \end{array}
\end{equation}

Its KKT conditions are:
\begin{equation}
  \label{KKT_equ}
  \left\{
\begin{array}{ll}
  \nabla_x\mathcal{L}(x^*,\lambda^*) = 0\\
  C(x^*) = 0\\
\end{array}
\right.
\end{equation}

We denote with a subscript $y_k$ the value of the quantity $y$ evaluated at point $(x_k, \lambda_k)$.

The first order linearization of \ref{KKT_equ} gives:

\begin{equation}
  \label{KKT_1st_order}
  \begin{array}{l}

  \left\{
\begin{array}{l}
  \nabla_{xx}^2\mathcal{L}_k z_x + \nabla_{x\lambda}^2\mathcal{L}_k z_\lambda + \nabla_x\mathcal{L}_k  = 0\\
  \nabla_x C_k z_x + C_k = 0\\
\end{array}
\right. \\
\text{which is equivalent to:}\\
  \left\{
\begin{array}{l}
  \nabla_{xx}^2\mathcal{L}_k z_x + \nabla_{x}C_k (\lambda_k + z_\lambda) = - \nabla_{x}f_k\\
  \nabla_x C_k z_x = - C_k \\
\end{array}
\right. \\
\text{Or in matrix form:}\\
  \begin{pmatrix}
      \nabla_{xx}^2\mathcal{L}_k & \nabla_x C_k\\
      \nabla_x C_k & 0\\
  \end{pmatrix}
  \begin{pmatrix}
      z_x\\
      \lambda_{k+1}\\
  \end{pmatrix}
  =
  \begin{pmatrix}
      - \nabla_{x}f_k\\
      - C_k\\
  \end{pmatrix}
  \end{array}
\end{equation}

Solving this problem actually is equivalent to solving a QP problem of the following form:

\begin{equation}
  \begin{array}{ll}
    \min\limits_{z_x} &f_k + \nabla_x f_k ^T z_x + \frac{1}{2} z_x^T\nabla_{xx}^2\mathcal{L}_k z_x \\
    \text{s.t.} & \nabla_x C_k^T z_x + C_k = 0\\
  \end{array}
\end{equation}

The unique solution of this problem satisfies the following matrix equality:

\begin{equation}
  \begin{pmatrix}
      \nabla_{xx}^2\mathcal{L}_k & -\nabla_x C_k\\
      \nabla_x C_k & 0\\
  \end{pmatrix}
  \begin{pmatrix}
      z_x\\
      l_k\\
  \end{pmatrix}
  =
  \begin{pmatrix}
      - \nabla_{x}f_k\\
      - C_k\\
  \end{pmatrix}
\end{equation}

Note that this problem has a solution if the following assumptions hold:
\begin{itemize}
  \item The constraint Jacobian $\nabla_x C_k$ has full rank.
  \item The matrix $\nabla_{xx}^2\mathcal{L}_k$ is positive definite on the tangent space of the constraints:\\ $d^T\nabla_{xx}^2\mathcal{L}_k d \geq 0,\ \forall d\neq 0$ such that ${\nabla_x C_k}^T d = 0$.
\end{itemize}

Therefore, the problem raised by the linearization of the KKT conditions to the first order can be solved as a QP problem.
Denoting the solution of the QP problem $(z_x, l_k)$, the solution to \ref{KKT_1st_order} is given by

\begin{equation}
  \begin{pmatrix}
      z_x\\
      \lambda_{k+1}\\
  \end{pmatrix}
  \leftarrow
  \begin{pmatrix}
      z_x\\
      -l_k\\
  \end{pmatrix}
\end{equation}

This development can easily be extended to treat the case of problems constrained with inequality constraints:

\begin{equation}
  \min_{x\in\mathbb{R}^n}{f(x)} \text{ subject to }
  \left\{
  \begin{array}{l}
    c_i(x) = 0,\ \forall i\in{E}\\
    c_i(x) \geq 0,\ \forall i\in{I}\\
  \end{array}
  \right.
\end{equation}

By solving the following Inequality-constrained Quadratic Problem(IQP) system at each iteration:

\begin{equation}
  \label{IQP}
  \begin{array}{ll}
    \min\limits_{z_x} &f_k + \nabla_x f_k ^T z_x + \frac{1}{2} z_x^T\nabla_{xx}^2\mathcal{L}_k z_x \\
    \text{s.t.} & \nabla_x {c_i}_k z_x + {c_i}_k = 0 ,\ \forall i\in E\\
                & \nabla_x {c_i}_k z_x + {c_i}_k \geq 0 ,\ \forall i\in I\\
  \end{array}
\end{equation}

The main difficulty in solving that QP \ref{IQP} comes from finding the optimal active set $\mathit{A}_k$.
It is a necessary but costly operation.
Especially when starting from a random active set $\mathit{A}_0$.
The process is mostly based on a try and guess approach guided by euristics(that may vary with different resolution algorithms).

To find the optimal active set, the QP starts from a random active set $\mathit{A}_0$ ignore the non-active constraints and try to solve the QP with the remaining active constraints.
At a step $k$ of the resolution of the IQP, all the constraints in the active set $\mathit{A}$ are treated as equality constraints and the others are simply ignored, the QP to solve becomes:

\begin{equation}
  \text{EQP} \equiv \left\{
  \begin{array}{ll}
    \min\limits_{z_x} &f_k + \nabla_x f_k ^T z_x + \frac{1}{2} z_x^T\nabla_{xx}^2\mathcal{L}_k z_x \\
    \text{s.t.} & \nabla_x {c_i}_k z_x + {c_i}_k = 0 ,\ \forall i\in \mathit{A}_k\\
  \end{array}
  \right.
\end{equation}

This EQP is solved, and its solution is checked against the constraints that were previously ignored.
If some of them are violated then the solution is not satisfactory and the active set is updated by adding one or several of the violated constraints to it.
This operation is repeated until there is no more constraints to be added.
Once all the constraints have been adde as active constraints to the EQP, the problem may be overconstrained.
To detect that, the Lagrange multipliers are computed and if some of them do not agree with the KKT conditions of the IQP \ref{IQP}, then one of them is removed from the active set and the resulting EQP is solved, and the process continues until a satisfactory solution is found.
The euristics used to choose which constraints to add or remove from the active set and to avoid cycling between active sets are out of the scope of this dissertation.

Once the SQP gets close to the solution, the set of active constraints should not change much from one iterate $x_k$ to the next one.
It is often useful to initialize the IQP's active set with the optimal active set of the previous iterate: $\mathit{A}_0(x_{k+1})\leftarrow\mathit{A}^*(x_k)$. Then the ${(k+1)}^{th}$ QP starts from an active set close to optimal and can be solved much faster.
That method is called the \textit{warm-start}.

The SQP approach to solving the problem \ref{formulation_NLCP} as presented above leans on the assumption that at each step, the QP generated is feasible,
which requires that the Hessian $\nabla_{xx}^2\mathcal{L}$ is positive definite on the tangent space of the active constraints.
That property is guarantied to hold near the solution, but not when starting from a remote point on a non-convex problem.
To quope with that issue, several methods exist.
In the next few section, we will present the globalisation methods that are the Line Search and Trust Region approaches, they ensure that the QP subproblem is always feasible.
Then we'll present some methods used to choose whether to accept of reject an iterate with the merit function and filter methods.
And finally we will discuss some Hessian approximation methods.

\subsection{Globalization methods}
\label{sub:globalization_methods}

As we stated above, the globalisation phase of an SQP optimization is meant to modify the step generated by the QP resolution in order to improve the the global convergence of the algorithm.
Either by multiplying the step found by the QP with a coefficient $\alpha$ in the Line-Search approach.
Or by bounding the norm of the step found in the QP by adding a boundary constraint on $z$ directly inside of the QP problem.

\subsubsection{The Line-Search Strategy}
In the Line-Search strategy, starting from a current iterate $x_k$ with a QP generated step $z$ solution of the linearized problem. We search a coefficient $\alpha$ such that $x_{k+1}=x_k+\alpha z$ is satisfactory.
A usual way to decide if a point is satisfactory is to use a Merit function.
A Merit function is somewhat similar to a penalty function.
In the sense that it estimates the quality of a point from the value of the cost function penalized by the violation of the constraints.
A typical merit function for a problem with only equality constraints can be written as:
\begin{equation}
  M(x,\mu) = f(x)+\mu \|c(x)\|_1
\end{equation}

If inequality constraints are present, a simple penalty function like \ref{penalty_function} would not be satisfactory because of the non-smoothness of the operator $[.]^-$.
To avoid that problem, slack variables can come in handy to replace the inequality constraints in the problem as follows:

\begin{equation}
  c_i(x)\geq 0\ \equiv
  \left\{
  \begin{array}{l}
    c_i(x)-s_i=0\\
    s_i\geq 0
  \end{array}
  \right.
\end{equation}


given a point $x_k$, a descent direction $p_k$ from this
point is chosen and then a length of step is calculated to minimize the
following 1-dimentional problem:
\begin{align}
  \min_{\alpha \in \mathbb{R}^+} f(x_k + \alpha.p_k)
\label{eq:lineSearch}
\end{align}

Once the best value of $\alpha$ has been found, the next iterate is computed:
$x_{k+1} = x_{k} + \alpha p_k$ and the same process is repeated until a
satisfying solution is found.

It is not always necessary to find the optimal value of $\alpha$, especially if
that is expensive. Indeed, $\alpha$ is only used to calculate the next iterate,
from which another $p_k$ and $\alpha$ will be calculated. And in the end, the
imprecision on the computation of $\alpha$ will be errased in the other steps of
the resolution.

There are several ways to choose a descent direction from an iterate $x_k$. The
most obvious one is probably the steepest descent direction $-\nabla f_k$. This
method provides the direction along which f decreases most rapidly and only
requires the evaluation of the first derivative of $f$, but that method can
become extremely slow on complicated problems. Another popular approach is the
Newton method, in which the objective function is approximated to the second
order

\begin{equation}
  f(x_k+p) = f_k + p^T\nabla f_k + p^T\nabla^2f_k p
\end{equation}

Then the chosen descent direction is the optimum of that approximated function,
the Newton direction.
\begin{equation}
  p^N_k = -(\nabla^2 f_k)^{-1} \nabla f_k
\end{equation}
The choice of this descent direction implies that the $\nabla^2f_k$ is positive
definite, in which case an adaptation of the definition of $p_k$ is required. Or
an approximation $B_k$ of $\nabla^2f_k$ that guaranties definite positiveness can be
used. For example, the symmetric-rank-one(SR1) formula and the BFGS(Broyden,
Fletcher, Goldfarb, Shanno) formula. Then the step becomes

\begin{equation}
  p_k = -B_k^{-1}\nabla f_k
\end{equation}

\subsubsection{The Trust-Region Strategy}
The Trust-Region Strategy works in an opposite way than the line-search one in
the sense that during a line-search step, a direction is chosen, and based on
that direction, a step-length is chosen. Whereas with a Trust-Region approach, a
maximum step-length is chosen, and based on it, the descent direction and lenght
are chosen.
The principle of the trust region is that along the optimization process, a
model of the problem is constructed and enriched at every step and at each step,
the next iterate is the optimum of the model, with the constraint that the step
to get there lies inside the trust-region. For example, let us considere that
the trust region is a sphere of center $x_k$ and radius $\rho_k$, then the
constraint on $p_k$ is $\|p_k\| \leq \rho_k$. A usual model to take for the
objective function is the quadratic model with approximated Hessian
\begin{equation}
  m_k(p) = f(x_k+p) = f_k + p^T\nabla f_k + p^TB_k p
\end{equation}
And the optimization problem to solve at each step of the optimization is

\begin{align}
  \min_{p} & \quad f_k + p^T\nabla f_k + p^TB_k p \nonumber\\
\text{s.t.}&
\quad \|p\| \leq \rho_k
\label{eq:trustRegion}
\end{align}

Once the solution $p_k$ to this quadratic problem is found, its quality is
estimated by evaluating the value of $f(x_k+p_k)$. If the actual decrease of $f$
is satisfying (compared to the decrease predicted by the model) then the step is
accepted and the size of the trust-region can be increased. Otherwise, the step
is refused, the trust-region radius is reduced and a new step from $x_k$ is
computed on that smaller trust region.

\subsection{Quasi-Newton Approximation}
\label{sub:quasi_newton_approximation}

In the SPQ algorithm, it is necessary to have access to the Hessian of the Lagrangian $\nabla_{xx}^2\mathcal{L}$ to be able to devise the QP subproblem to solve.
For some strategies like the Line-Search, it is necessary that $\nabla_{xx}^2\mathcal{L}$ is definite positive.
Sometimes the exact Hessian of the problem is not positive definite.
Also it is often difficult or computationally expensive to compute an exact Hessian of the Lagrangian.
Since we are following an iterative process, it is not necessary to have an exact knowledge of the Hessian and using an approximation of it is usually enough.
Also, an approximate Hessian is in most cases less expensive to compute than the exact one.

The idea behind computing an approximate Hessian is that, starting from an initial approximate Hessian $B_0$, we compute at each iteration an update to the approximate Hessian based on the values and first order derivatives of the Lagrangian.
This update aims at capturing some curvature information about the Hessian by evaluating the evolution of the gradient along the latest step.
At step $k$, the Hessian update is a function of $s_k$ and $y_k$:
\begin{equation}
  s_k = x_{k+1}-x_k,\ \ \ \
  y_k = \nabla_x\mathcal{L}(x_{k+1}, \lambda_{k+1}) - \nabla_x\mathcal{L}(x_{k}, \lambda_{k+1})
\end{equation}

The two most famous Hessian update strategies are called the BFGS (Broyden–Fletcher– Goldfarb–Shanno) and the SR1(Symmetric Rank 1) updates.
BFGS is a rank 2 update while SR1 is rank 1.

The most basic formulas for the BFGS update is the following:

\begin{equation}
  \label{BFGS}
  B_{k+1} = B_k - \frac{B_k s_k s_k^T B_k}{s_k^T B_k s_k} + \frac{y_k y_k^T}{s_k^T y_k}
\end{equation}

Note that using a BFGS update requires that $s_k$ and $y_k$ satisfy the curvature condition: $s_k^Ty_k>0$.
If that condition does not hold, then the value of $y_k$ is modified, wich gives rise to the Damped BFGS update, which guarantees to keep $B_k$ definite positive:

\begin{equation}
  \label{damped_BFGS}
\begin{split}
  \theta_k =
  \left\{
      \begin{array}{ll}
      1 & \text{if } s_k^Ty_k \geq 0.2s_k^TB_ks_k \\
      \frac{0.8s_k^TB_ks_k}{s_k^TB_ks_k-s_k^Ty_k} & \text{if } s_k^Ty_k \geq 0.2s_k^TB_ks_k \\
      \end{array}
      \right.\\
      r_k = \theta_k y_k + (1-\theta_k)B_ks_k\\
      B_{k+1} = B_k - \frac{B_k s_k s_k^T B_k}{s_k^T B_k s_k} + \frac{r_k r_k^T}{s_k^T r_k}
\end{split}
\end{equation}

The SR1 update is computed with the following formula:

\begin{equation}
  \label{SR1}
  B_{k+1} = B_k + \frac{(y_k-B_ks_k)(y_k-B_ks_k)^T}{(y_k-B_ks_k)^Ts_k}
\end{equation}

Both those formulas proved to be efficient in some cases. It is not yet clear in which cases one is better than the other.

\section{Conclusion}
This section gives a general introduction to Non-linear constrained optimization without regards for robotics or optimization on manifolds

