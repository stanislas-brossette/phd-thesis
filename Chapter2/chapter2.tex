\chapter{
Posture Generation for Humanoid Robotic Systems
}

\label{chapter:PG}

\ifpdf
    \graphicspath{{Chapter2/Figs/Raster/}{Chapter2/Figs/PDF/}{Chapter2/Figs/}}
\else
    \graphicspath{{Chapter2/Figs/Vector/}{Chapter2/Figs/}}
\fi


%%%%%%%%%%%%%%%%%%%%%%%%%%
%  SECTION INTRODUCTION  %
%%%%%%%%%%%%%%%%%%%%%%%%%%

\section{Introduction}

Posture Generation is a necessary tool that provides the key postures for many robotics applications like planning, trajectory generation, control.

More explanation about why it is necessary.

%%%%%%%%%%%%%%%%%%%%%%%%%
%  SECTION FORMULATION  %
%%%%%%%%%%%%%%%%%%%%%%%%%

\section{Formulation}

Here we present the basic formulation of a PG problem.

The free-flyer, the articular space and the forces being the usual variables.

Presentation of all the constraints and their formulation
\begin{itemize}
  \item Joint limits
  \item Auto-collision
  \item Collisions with environment
  \item Contacts with environment
  \item Stability
  \item Torque limits
\end{itemize}

%%%%%%%%%%%%%%%%%%%%%%%%%%%%%%%%%%%%%%%%%%%%%%%%%%%%%%%%%%%%%%%%%
%  SECTION APPLICATION TO CONTACT PLANNING ON REAL ENVIRONMENT  %
%%%%%%%%%%%%%%%%%%%%%%%%%%%%%%%%%%%%%%%%%%%%%%%%%%%%%%%%%%%%%%%%%

\section{Application to contact planning on real environment}
\subsection{Understanding the environment through a point-cloud}
Segmentation pipeline:
\begin{itemize}
  \item Acquisition
  \item Filtering
  \item Region growing segmentation
  \item Planar Extraction
  \item Planar projection and Hull convex generation
  \item Re-orientation and transfer to the planner
\end{itemize}

\subsection{Contact generation with convex surface inclusions}
Usual methods for generating surface contact are based on point-to-point sampling, on rectangular inclusion, or other limiting methods. We proposed to extend that to the inclusion of convex surfaces.

\subsection{Simulated scenarios planned on real-environment}
We present 2 planned scenarios with the HRP-2 robot



%%%%%%%%%%%%%%%%%%%%%%%%%%%%%%%%%%%%%%%%%%%%%%%%%%%%%%%%%%%%%%%%%%%%%%%%%
%  SECTION INTEGRATION ON NON-INCLUSIVE CONTACTS IN POSTURE GENERATION  %
%%%%%%%%%%%%%%%%%%%%%%%%%%%%%%%%%%%%%%%%%%%%%%%%%%%%%%%%%%%%%%%%%%%%%%%%%

\section{Integration on Non-Inclusive Contacts in Posture Generation}

The method presented before is still limiting in many cases. Ladder, stairs climbing\dots
Sometimes it is not possible to ensure inclusion of 2 surfaces.

\subsection{Contact geometry formulation}
\label{sec:background}
Present the difficulties related to intersecting 2 convex polygons. Non-constant number of constraints.
Usual solution: patching

\subsection{Non inclusive contact constraints}
\subsubsection{Main Idea}
\subsubsection{Pseudo-distance}
\subsubsection{Modification of the optimization problem}
\subsubsection{Maximixation of the contact area}
\subsubsection{Using non inclusuve contact to maintain stability}
\subsubsection{Extension to singular cases}
\subsection{Simulation Results}
\subsubsection{Inclined ladder climbing}
\subsubsection{Vertical ladder climbing}
\subsubsection{Climbing Stairs}
\subsubsection{Walking along a path made of small objects}

%%%%%%%%%%%%%%%%%%%%%%%%
%  SECTION CONCLUSION  %
%%%%%%%%%%%%%%%%%%%%%%%%

\section{Discussion and conclusion}
All that is very nice, but still many limitations and possible improvements.
Transition to Posture generation on Non-Euclidean Manifolds.
