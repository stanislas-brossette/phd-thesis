
%%%%%%%%%%%%%%%%%%%%%%%%%%%%%%%%%%%%%%%%%%%%%%%%%%%%%%%%%%%%%%%%%%%%%%%
%                           Second Chapter                            %
%                  Extensions of Posture Generation                   %
%%%%%%%%%%%%%%%%%%%%%%%%%%%%%%%%%%%%%%%%%%%%%%%%%%%%%%%%%%%%%%%%%%%%%%%

\chapter{Extensions of Posture Generation}
\label{cha:extensions_of_posture_generation}

\section{List of contributions}
\paragraph{Integration on Non-Inclusive Contacts in Posture Generation}

The method presented before is limiting in many cases. Ladder, stairs climbing\dots.
Sometimes it is not possible to ensure inclusion of 2 surfaces.

\begin{itemize}
  \item Contact geometry formulation
    \begin{itemize}
      \item Discretisation of contact surfaces for practical reasons
      \item Contact generation with convex surface inclusions
Usual methods for generating surface contact are based on point-to-point sampling, on rectangular inclusion, or other limiting methods. We extend that to the inclusion of convex surfaces.
      \item Inclusive contacts are no problem
      \item Non-Inclusive contacts lead to non-constant number of constraints or non-smooth gradients, cannot be solved with usual solvers with that formulation
    \end{itemize}
  \item{Non inclusive contact constraints}
    \begin{itemize}
      \item {Main Idea: Inserting an ellipse in the intersection of polygons}
      \item {Pseudo-distance(to simplify the formulation)}
      \item {Modification of the optimization problem}
      \item {Maximization of the contact area}
      \item {Using non inclusive contact to maintain stability}
      \item {Extension to singular cases}
    \end{itemize}
  \item{Simulation Results}
    \begin{itemize}
      \item{Inclined ladder climbing}
      \item{Vertical ladder climbing}
      \item{Climbing Stairs}
      \item{Walking along a path made of small objects}
    \end{itemize}
  \item{Application to ladder climbing (cf. Joris papers)}
\end{itemize}

\paragraph{On the use of lifted variables for Robotics Posture Generation}
\begin{itemize}
  \item {Introduction: "J. Albersmeyer and M. Diehl, The lifted Newton Method and its Application in Optimisation"}
  \item {Lifting Algorithm: Introduce additional variables to reduce the complexity/degree of the equations to solve. And then use a trick to re-condense the system and avoid loosing computation time}
  \item {Optimization on lifted variables}
  \item {Condensed BFGS update}
  \item {Results, experimentation}
  \begin{itemize}
    \item Symbolic Inverse Kinematics
    \item Automatic Lifting algorithm adapted to posture generation
    \item Several Solvers for lifted and non-lifted systems (Gauss Newton, SQP, lifted Newton.... )
    \item Compared many optimization methods (Globalization, regularization....)
    \item Never managed to get results faster than basic methods
  \end{itemize}
  \item{Motivation:Development of a posture generator with a specialized solver}
    \begin{itemize}
      \item Problem
        \begin{itemize}
          \item Robotics problems solved with generic solvers (IPOPT, CFSQP, fmincon...)
          \item Robotics problems(especially posture generation) are very specific (Constraints, costs, working spaces, sizes...)
        \end{itemize}
      \item Solution
        \begin{itemize}
          \item Develop a posture generator, along with a non-linear solver
          \item Better understanding of our own solver
          \item Ability to use the solver in new ways:
          \begin{itemize}
            \item Working on Manifolds
            \item Variable numbers of constraints
            \item Automatic management of variables, constraints and mathematical expressions
          \end{itemize}
        \end{itemize}
    \end{itemize}
\end{itemize}

%%%%%%%%%%%%%%%%%%%%%%%%%%%%%%%%%%%%%%%%%%%%%%%%%%%%%%%%%%%%%%%%%
%  SECTION APPLICATION TO CONTACT PLANNING ON REAL ENVIRONMENT  %
%%%%%%%%%%%%%%%%%%%%%%%%%%%%%%%%%%%%%%%%%%%%%%%%%%%%%%%%%%%%%%%%%

\section{Application to contact planning on real environment}
\subsection{Understanding the environment through a point-cloud}
Segmentation pipeline:
\begin{itemize}
  \item Acquisition
  \item Filtering
  \item Region growing segmentation
  \item Planar Extraction
  \item Planar projection and Hull convex generation
  \item Re-orientation and transfer to the planner
\end{itemize}

\subsection{Contact generation with convex surface inclusions}
Usual methods for generating surface contact are based on point-to-point sampling, on rectangular inclusion, or other limiting methods. We proposed to extend that to the inclusion of convex surfaces.

\subsection{Simulated scenarios planned on real-environment}
We present 2 planned scenarios with the HRP-2 robot



%%%%%%%%%%%%%%%%%%%%%%%%%%%%%%%%%%%%%%%%%%%%%%%%%%%%%%%%%%%%%%%%%%%%%%%%%
%  SECTION INTEGRATION ON NON-INCLUSIVE CONTACTS IN POSTURE GENERATION  %
%%%%%%%%%%%%%%%%%%%%%%%%%%%%%%%%%%%%%%%%%%%%%%%%%%%%%%%%%%%%%%%%%%%%%%%%%

\section{Integration on Non-Inclusive Contacts in Posture Generation}

The method presented before is still limiting in many cases. Ladder, stairs climbing\dots.
Sometimes it is not possible to ensure inclusion of 2 surfaces.

\subsection{Contact geometry formulation}
\label{sec:background}
Present the difficulties related to intersecting 2 convex polygons. Non-constant number of constraints.
Usual solution: patching

\subsection{Non inclusive contact constraints}
\subsubsection{Main Idea}
\subsubsection{Pseudo-distance}
\subsubsection{Modification of the optimization problem}
\subsubsection{Maximixation of the contact area}
\subsubsection{Using non inclusuve contact to maintain stability}
\subsubsection{Extension to singular cases}
\subsection{Simulation Results}
\subsubsection{Inclined ladder climbing}
\subsubsection{Vertical ladder climbing}
\subsubsection{Climbing Stairs}
\subsubsection{Walking along a path made of small objects}

%%%%%%%%%%%%%%%%%%%%%%%%%%%%%%%%%%%%%%%%%%%%%%%%%%%%%%%%%%%%%%%%%%%%
%  ON THE USE OF LIFTED VARIABLES FOR ROBOTICS POSTURE GENERATION  %
%%%%%%%%%%%%%%%%%%%%%%%%%%%%%%%%%%%%%%%%%%%%%%%%%%%%%%%%%%%%%%%%%%%%

\section{On the use of lifted variables for Robotics Posture Generation}
\label{sec:liftedVariables}

\subsection{Introduction}
\label{subsec:Introduction}

\subsection{Lifting Algorithm}
\label{subsec:LiftingAlgorithm}

\subsection{Optimization on lifted variables}
\label{subsec:Optimization on lifted variables}

\subsection{Condensed BFGS update}
\label{subsec:Condensed BFGS update}

\subsection{Results, experimentation}
\label{subsec:Results}

\subsection{conclusion}
\label{subsec:conclusion}

%%%%%%%%%%%%%%%%%%%%%%%%
%  SECTION CONCLUSION  %
%%%%%%%%%%%%%%%%%%%%%%%%

\section{Discussion and conclusion}
All that is very nice, but still many limitations and possible improvements.
Transition to Posture generation on Non-Euclidean Manifolds.



