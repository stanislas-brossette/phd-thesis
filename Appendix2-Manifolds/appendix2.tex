
%%%%%%%%%%%%%%%%%%%%%%%%%%%%%%%%%%%%%%%%%%%%%%%%%%%%%%%%%%%%%%%%%%%%%%%
%                          Thesis Appendix B                          %
%                       Manifolds Descriptions                        %
%%%%%%%%%%%%%%%%%%%%%%%%%%%%%%%%%%%%%%%%%%%%%%%%%%%%%%%%%%%%%%%%%%%%%%%

\chapter{Manifolds Descriptions}
\label{appendix:manifolds}

\graphicspath{{Appendix2-Manifolds/Figs/}}

This appendix chapter comes as a complement for Chapter~\ref{chapter:optimization_on_noneuclidean_manifolds} in which we explicit the elements necessary for the description of several elementary manifolds.

%{{{ THE REAL SPACE MANIFOLD $\MATHBB{R^N$}
\section{The Real Space manifold}
\label{sec:the_real_space}
The Realspace manifold of dimension $n$ is denoted $\mathbb{R}^n$.
Since $\mathbb{R}^n$ is a Euclidean manifold, the operations that we use on it are straightforward.

\begin{table} [H]
\caption{Description of the $\mathbb{R}^n$ manifold}
\centering
\begin{tabular}{cc}
  \toprule
  $\mathcal{M}$ & $\mathbb{R}^n$ \\
  \midrule
  $\mathbb{E}$ & $\mathbb{R}^n$ \\
  \midrule
  $T_x\mathcal{M}$ & $\mathbb{R}^n$ \\
  \midrule
  $T_x\mathbb{E}$ & $\mathbb{R}^n$ \\
  \midrule
  $\phi_x(\mathbf{z})$ & $\mathbf{x} + \mathbf{z}$ \\
  \midrule
  $\partial \phi_x(0)$ & $\mathbb{I}_n$ \\
  \bottomrule
\end{tabular}
\quad
\begin{tabular}{cc}
  \toprule
  $\zeta(x,y)$ & $\mathbf{y} - \mathbf{x}$ \\
  \midrule
  $\frac{\partial \zeta_x}{\partial y}(x)$ & $\mathbb{I}_n$ \\
  \midrule
  $\mathcal{T}(x,\mathbf{z}, \mathbf{v})$ & $\mathbf{v}$ \\
  \midrule
  $\pi_\mathcal{M}(\mathbf{x})$ & $\mathbf{x}$ \\
  \midrule
  $\pi_{T_x\mathcal{M}}(\mathbf{z})$ & $\mathbf{z}$ \\
  \midrule
  $\lim$ & $\|\mathbf{v}\| \leq \infty$ \\
  \bottomrule
\end{tabular}
\end{table}
%}}}
%{{{ THE 3D ROTATION MANIFOLD SO(3): MATRIX REPRESENTATION
\section{The 3D Rotation manifold: Matrix representation}
\label{sec:the_3d_rotation_manifold_matrix_representation}

The 3D rotation manifold is denoted $SO(3)$.

An element $x$ of $SO(3)$ is represented by $\mathbf{x}=\psi(x)$ in $\mathbb{R}^{3\times 3}$ by:
\begin{equation}
  x\in SO(3),\ \mathbf{x} =\begin{bmatrix}
    x_{00} & x_{01} & x_{02} \\
    x_{10} & x_{11} & x_{12} \\
    x_{20} & x_{21} & x_{22} \\
  \end{bmatrix}
\end{equation}

We recall the operators
\begin{equation}
\widehat{.}: \begin{bmatrix}
  \omega_0\\\omega_1\\\omega_2\\
\end{bmatrix}
\rightarrow
\begin{bmatrix}
  0 & -\omega_2 & \omega_1 \\
  \omega_2 & 0 & -\omega_0 \\
  -\omega_1 & \omega_0 & 0\\
\end{bmatrix}
\end{equation}
And its inverse:
\begin{equation}
\widecheck{.}: \begin{bmatrix}
    x_{00} & x_{01} & x_{02} \\
    x_{10} & x_{11} & x_{12} \\
    x_{20} & x_{21} & x_{22} \\
\end{bmatrix}
\rightarrow
\begin{bmatrix}
  x_{21}\\x_{02}\\x_{10}\\
\end{bmatrix}
\end{equation}


The exponential map is known as the Rodrigues formula:
\begin{equation}
  \forall \mathbf{v}\in\mathbb{R}^3,\ \exp(\mathbf{v}) = \mathbb{I}_3 +
  \frac{\sin \|\mathbf{v}\|}{\|\mathbf{v}\|} \hat{\mathbf{v}} +
  \frac{1-\cos \|\mathbf{v}\|}{\|\mathbf{v}\|^2} \hat{\mathbf{v}}^2
\end{equation}

Note that when $\|\mathbf{v}\|$ is small, we make the following replacements to avoid numerical instability.
It is important to ensure the precision of the retractation near zero because, in an optimization process, many small steps are taken, especially when close to the solution.

\begin{align}
 \frac{\sin \|\mathbf{v}\|}{\|\mathbf{v}\|} & = 1-\frac{\|\mathbf{v}\|}{6}\\
 \frac{1-\cos \|\mathbf{v}\|}{\|\mathbf{v}\|^2} & = 0.5 - \frac{\|\mathbf{v}\|}{24}
\end{align}

And the logarithm is computed as follows (see~\cite{merlhiot:thesis:2009}):
\begin{align}
\begin{split}
  \forall R\in\psi(SO(3)),\ f(R) &=
  \left\{ \begin{matrix}
  0 & \text{if }Tr(R) = 3 \\
  \frac{\arccos\left(\frac{Tr(R)-1}{2}\right)}{2\sin\left(\arccos\left(\frac{Tr(R)-1}{2}\right)\right)}\left(R-R^T\right) & \text{otherwise} \\
  \end{matrix} \right.\\
  \log(R) &= \widecheck{f\left(R\right)}
\end{split}
\end{align}

We consider a point $x\in\mathcal{M}$ and its representation matrix with the following storing order:
\begin{equation}
\mathbf{x}=\psi(x)
= \begin{pmatrix}
  x_0 & x_3 & x_6 \\
  x_1 & x_4 & x_7 \\
  x_2 & x_5 & x_8 \\
\end{pmatrix}
= \begin{pmatrix}
  x_{00} & x_{01} & x_{02} \\
  x_{10} & x_{11} & x_{12} \\
  x_{20} & x_{21} & x_{22} \\
\end{pmatrix}
\end{equation}

The derivative of retractation operation writes as:

\begin{align}
\label{eq:diffRetrSO3Matrix}
  \frac{\partial \varphi_x}{\partial \mathbf{z}}(0) =
  \begin{bmatrix}
    0 & -x_6 & x_3 \\
    0 & -x_7 & x_4 \\
    0 & -x_8 & x_5 \\
    x_6 & 0 & -x_0 \\
    x_7 & 0 & -x_1 \\
    x_8 & 0 & -x_2 \\
    -x_3 & x_0 & 0 \\
    -x_4 & x_1 & 0 \\
    -x_5 & x_2 & 0 \\
  \end{bmatrix}
\end{align}

And the derivation of the logarithm operation writes as:

\begin{align}
\label{eq:diffLogSO3Matrix}
  \mathbf{v} &= \begin{pmatrix}
    \frac{x_{21} - x_{12}}{2}\\
    \frac{x_{02} - x_{20}}{2}\\
    \frac{x_{10} - x_{01}}{2}\\
  \end{pmatrix} \\
  f &= \frac{\arccos \left( \frac{Tr(\mathbf{x})-1}{2} \right)}{2 \sin \left( \arccos \left( \frac{Tr(\mathbf{x})-1}{2} \right) \right) } \\
  df &= \frac{\left(Tr(\mathbf{x})-1\right)f-1}{2 \left( 1- {\left( \frac{Tr(\mathbf{x})-1}{2} \right)}^2 \right)} \\
  \frac{\partial \zeta_x}{\partial y}(x) &= \begin{bmatrix}
      & 0 & 0 & 0 &  & f & 0 & -f &  \\
    df.\mathbf{v} & 0 & -f & 0 & df.\mathbf{v} & 0 & f & 0 & df.\mathbf{v}  \\
      & f & 0 & -f &  & 0 & 0 & 0 &  \\
  \end{bmatrix}
\end{align}

\begin{table} [H]
\caption{Description of the $SO(3)$ manifold with matrix representation}
\centering
\begin{tabular}{cc}
  \toprule
  $\mathcal{M}$ & $SO(3)$ \\
  \midrule
  $\mathbb{E}$ & $\mathbb{R}^{3\times 3}$ \\
  \midrule
  $T_x\mathcal{M}$ & $\mathbb{R}^3$ \\
  \midrule
  $T_x\mathbb{E}$ & $\mathbb{R}^3$ \\
  \midrule
  $\psi(x) = \mathbf{x}$ & $ \begin{bmatrix}
    x_{00} & x_{01} & x_{02} \\
    x_{10} & x_{11} & x_{12} \\
    x_{20} & x_{21} & x_{22} \\
  \end{bmatrix} $ \\
  \midrule
  $\phi_x(\mathbf{z})$ & $\mathbf{x}\exp(\mathbf{z})$ \\
  \bottomrule
\end{tabular}
\quad
\begin{tabular}{cc}
  \toprule
  $\partial \phi_x(0)$ & see Equation~\ref{eq:diffRetrSO3Matrix} \\
  \midrule
  $\zeta(x,y)$ & $\log(\mathbf{x}^T\mathbf{y})$ \\
  \midrule
  $\frac{\partial \zeta_x}{\partial y}(x)$ & see Equation~\ref{eq:diffLogSO3Matrix} \\
  \midrule
  $\mathcal{T}(x,\mathbf{z}, \mathbf{v})$ & $\mathbf{v}$ \\
  \midrule
  $\pi_\mathcal{M}(\mathbf{x})$ & Q from QR decomposition of $\mathbf{x}$ \\
  \midrule
  $\pi_{T_x\mathcal{M}}(\mathbf{z})$ & $\mathbf{z}$ \\
  \midrule
  $\lim$ & $\|\mathbf{v}\| \leq \pi$ \\
  \bottomrule
\end{tabular}
\end{table}

%}}}

%{{{ THE 3D ROTATION MANIFOLD SO3 QUATERNION REPRESENTATION
\section{The 3D Rotation manifold with quaternion representation}
\label{sec:the_3d_rotation_manifold_quaternion_representation}

The 3D rotation manifold is denoted $SO(3)$.

An element $x$ of $SO(3)$ is represented by $\mathbf{q}=\psi(x)$ in $\mathbb{R}^{4}$ by:
\begin{equation}
  x\in SO(3),\ \mathbf{q} =\begin{bmatrix}
    q_{w}\\
    q_{x}\\
    q_{y}\\
    q_{z}\\
  \end{bmatrix}
  =\begin{bmatrix}
    q_{w}\\
    \mathbf{q_{vec}}\\
  \end{bmatrix}
\end{equation}

The exponential map is:
\begin{equation}
  \exp\ :\left|
  \begin{array}{ccc}
    \mathbb{R}^3 & \rightarrow & \mathbb{R}^4 \\
    \mathbf{z} & \mapsto & \begin{bmatrix}
      \cos \left( \frac{\|\mathbf{z}\|}{2} \right)\\
      \sin \left( \frac{\|\mathbf{z}\|}{2} \right) \frac{\mathbf{z}}{\|\mathbf{z}\|}\\
    \end{bmatrix} \\
  \end{array} \nonumber%
  \right.
\end{equation}

Note that when $\|\mathbf{v}\|$ is small, we make the following replacements to avoid numerical instability.

\begin{equation}
  \exp(\mathbf{z}) = \begin{bmatrix}
    1 -\frac{\|\mathbf{z}\|}{8} + \frac{{\|\mathbf{z}\|}^2}{384}\\
    \left(0.5 - \frac{\|\mathbf{z}\|}{48} + \frac{{\|\mathbf{z}\|}^2}{3840}\right)\mathbf{z}
  \end{bmatrix}
\end{equation}

And the logarithm is:
\begin{equation}
  \log\ :\left|
  \begin{array}{ccc}
    \mathbb{R}^4 & \rightarrow & \mathbb{R}^3 \\
    q & \mapsto & \arctan \left( \frac{2 \|\mathbf{q_{vec}}\| q_w}{q_w^2 - {\|\mathbf{q_{vec}\|}^2}} \right) \frac{\mathbf{q_{vec} } }{\|\mathbf{q_{vec}}\|} \\
  \end{array}
  \right.
\end{equation}

We consider a point $q\in\mathcal{M}$ and its representation quaternion with the following storing order:
\begin{equation}
\mathbf{q}=\psi(q)
= \begin{pmatrix}
  q_w \\
  q_x \\
  q_y \\
  q_z \\
\end{pmatrix}
\end{equation}

The derivative of retractation operation writes as:

\begin{align}
\label{eq:diffRetrSO3Quat}
  \frac{\partial \varphi_q}{\partial \mathbf{z}}(0) = \frac{1}{2}
  \begin{bmatrix}
    -q_x & -q_y & -q_z \\
     q_w & -q_z &  q_y \\
     q_z &  q_w & -q_x \\
    -q_y &  q_x &  q_w \\
  \end{bmatrix}
\end{align}

And the derivation of the logarithm operation writes as:

\begin{align}
\label{eq:diffLogSO3Quat}
  f &= \frac{\arctan \left( \frac{2 \|\mathbf{q_{vec}}\| q_w}{q_w^2 - {\|\mathbf{q_{vec}}\|}^2} \right)}{n}\\
  g &= \frac{2 q_w - f}{{\|\mathbf{q_{vec}}\|}^2} \\
  \frac{\partial \zeta_x}{\partial y}(\mathbf{x}) &= \begin{bmatrix}
  \begin{pmatrix}
    \\
    -2\mathbf{q_{vec}}\\
    \\
  \end{pmatrix} &
  \begin{pmatrix}
    \\
    f\mathbb{I}_3 + g\mathbf{q_{vec}}\mathbf{q_{vec}}^T\\
    \\
  \end{pmatrix}
  \end{bmatrix}
\end{align}


\begin{table} [H]
\caption{Description of the $\mathbf{SO(3)}$ manifold with quaternion representation}
\centering
\begin{tabular}{cc}
  \toprule
  $\mathcal{M}$ & $SO(3)$ \\
  \midrule
  $\mathbb{E}$ & $\mathbb{R}^{4}$ \\
  \midrule
  $T_x\mathcal{M}$ & $\mathbb{R}^3$ \\
  \midrule
  $T_x\mathbb{E}$ & $\mathbb{R}^3$ \\
  \midrule
  $\phi_x(\mathbf{z})$ & $\mathbf{x}\exp(\mathbf{z})$ \\
  \midrule
  $\partial \phi_x(0)$ & see Appendix~\ref{eq:diffRetrSO3Quat} \\
  \bottomrule
\end{tabular}
\quad
\begin{tabular}{cc}
  \toprule
  $\zeta(x,y)$ & $\log(\mathbf{x}^{-1}\mathbf{y})$ \\
  \midrule
  $\frac{\partial \zeta_x}{\partial y}(x)$ & see Appendix~\ref{eq:diffLogSO3Quat} \\
  \midrule
  $\mathcal{T}(x,\mathbf{z}, \mathbf{v})$ & $\mathbf{v}$ \\
  \midrule
  $\pi_\mathcal{M}(\mathbf{x})$ & $\frac{\mathbf{x}}{\|\mathbf{x}\|}$ \\
  \midrule
  $\pi_{T_x\mathcal{M}}(\mathbf{z})$ & $\mathbf{z}$ \\
  \midrule
  $\lim$ & $\|\mathbf{v}\| \leq \pi$ \\
  \bottomrule
\end{tabular}
\end{table}

%}}}


%{{{ THE UNIT SPHERE MANIFOLD $S^2$
\section{The Unit Sphere manifold}
\label{sec:the_unit_sphere_manifold_s2}

An element $x$ of $S^2$ is represented by $\mathbf{x}=\psi(x)$ in $\mathbb{R}^{3}$ by:
\begin{equation}
  x\in S^2,\ \mathbf{x} =\begin{bmatrix}
    x_0\\
    x_1\\
    x_2\\
  \end{bmatrix}
\end{equation}

With this manifold, the tangent space at $x$ is the tangent plane to the unit-sphere at $x$.
Thus, $T_x\mathcal{M}$ it is a 2-dimensional space, and its representation space is $\mathbb{R}^3$.
%A tangent vector to $x$, $\mathbf{z}$, is such that $\mathbf{x}\cdot \mathbf{z} = \mathbf{x}^T\mathbf{z}=0$.

For the retractation we simply use a normalized sum:
\begin{equation}
  \phi_x(\mathbf{z}) = \frac{\mathbf{x}+\mathbf{z}}{\|\mathbf{x}+\mathbf{z}\|}
\end{equation}

We define a distance operation on $S^2$ as:
\begin{equation}
  \dist(x, y) = 1-\mathbf{x}\cdot \mathbf{y}
\end{equation}

And the projection on the tangent space:
\begin{equation}
  \pi_{T_x\mathcal{M}}(\mathbf{z}) = \mathbf{z} - (\mathbf{x} \cdot \mathbf{z}) \mathbf{x}
\end{equation}

The pseudo-logarithm operation is the following:
\begin{equation}
  \zeta_x(y) = \dist(x,y)\frac{\pi_{T_x\mathcal{M}}(\mathbf{z})}{\|\pi_{T_x\mathcal{M}}(\mathbf{z})\|}
\end{equation}

The vector transport operation of vector $\mathbf{v}$ from $T_x\mathcal{M}$ to $T_y\mathcal{M}$ with $y = \phi_x(\mathbf{v})$, corresponds to rotating $\mathbf{v}$ by the rotation that transforms $x$ into $y$:
\begin{align}
  &\mathbf{y} = \phi_x(\mathbf{z}) \\
  &R = \mathbb{I}_3 + \widehat{\mathbf{x} \wedge \mathbf{y}} + \frac{{\widehat{\mathbf{x} \wedge \mathbf{y} } }^2}{1+\mathbf{x}\cdot\mathbf{y}} \\
  &\mathcal{T}(x,\mathbf{z}, \mathbf{v}) = R\mathbf{v}
\end{align}

\begin{table} [H]
\caption{Description of the $S^2$ manifold}
\centering
\begin{tabular}{cc}
  \toprule
  $\mathcal{M}$ & $S^2$ \\
  \midrule
  $\mathbb{E}$ & $\mathbb{R}^{3}$ \\
  \midrule
  $T_x\mathcal{M}$ & $\mathbb{R}^2$ \\
  \midrule
  $T_x\mathbb{E}$ & $\mathbb{R}^3$ \\
  \midrule
  $\phi_x(\mathbf{z})$ & $\frac{\mathbf{x}+\mathbf{z}}{\|\mathbf{x}+\mathbf{z}\|}$ \\
  \midrule
  $\partial \phi_x(0)$ & $\mathbb{I}_3 - \mathbf{x}\cdot\mathbf{x}^T$ \\
  \bottomrule
\end{tabular}
\quad
\begin{tabular}{cc}
  \toprule
  $\zeta(x,y)$ & $\mathbf{y} - (\mathbf{x} \cdot \mathbf{y}) \mathbf{x}$ \\
  \midrule
  $\frac{\partial \zeta_x(x)}{\partial y}$ & $\mathbb{I}_3 -\mathbf{x}\cdot\mathbf{x}^T$ \\
  \midrule
  $\mathcal{T}(x,\mathbf{z}, \mathbf{v})$ & $\mathbb{I}_3 + \widehat{\mathbf{x} \wedge \phi_x(\mathbf{z})} + \frac{{\widehat{\mathbf{x} \wedge \phi_x(\mathbf{z})}}^2}{1+\mathbf{x}\cdot\phi_x(\mathbf{z})}$ \\
  \midrule
  $\pi_\mathcal{M}(\mathbf{x})$ & $\frac{\mathbf{x}}{\|\mathbf{x}\|}$ \\
  \midrule
  $\pi_{T_x\mathcal{M}}(\mathbf{z})$ & $\mathbf{z} - (\mathbf{x} \cdot \mathbf{z}) \mathbf{x}$ \\
  \midrule
  $\lim$ & $\|\mathbf{v}\| \leq \infty$ \\
  \bottomrule
\end{tabular}
\end{table}

%}}}
