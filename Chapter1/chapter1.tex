%*******************************************************************************
%*********************************** First Chapter *****************************
%*******************************************************************************

\chapter{Posture Generation, state of the Art, Introduction}

\nomenclature[z-SQP]{SQP}{Sequential Quadratic Programing}
\nomenclature[z-FEM]{FEM}{Finite Element Method}


\ifpdf
    \graphicspath{{Chapter1/Figs/Raster/}{Chapter1/Figs/PDF/}{Chapter1/Figs/}}
\else
    \graphicspath{{Chapter1/Figs/Vector/}{Chapter1/Figs/}}
\fi

\section{List of contributions}
\begin{itemize}
  \item Generalities, introduction
  \item Presentation of the existing methods
  \item From Inverse Kinematics to Generalized IK/posture Generation/pose estimation (addition of articular limits, forces, stability etc.
  \item Topology of the parametrization space (Free-flyer, q, f, other)
  \item Formulation as a non-linear constrained optimization problem
  \item Adrien \& Karim's formulations
  \item Formulation of several types of cost/constraints
  \begin{itemize}
    \item Contact with plane surface
    \item Collision avoidance
    \item Auto-Collision avoidance
    \item Static equilibrium: Newton/CoM projection
    \item Forces in friction cones
    \item Articular limits
    \item Torque limits
    \item Torque minimization
    \item Goal Posture
  \end{itemize}
  \item Reasons why it is not enough and why we needed a new PG
    \begin{itemize}
      \item Having an easier way to formulate problems
      \item Avoid having to de some gymnastic to remain on manifolds
      \item Automatic variable management
      \item Robustness
    \end{itemize}
  \item Utilization of posture generation in planning
\end{itemize}


%%%%%%%%%%%%%%%%%%%%%%%%%%
%  SECTION INTRODUCTION  %
%%%%%%%%%%%%%%%%%%%%%%%%%%

\section{Introduction}

Posture Generation is a necessary tool that provides the key postures for many robotics applications like planning, trajectory generation, control.

More explanation about why it is necessary.

%%%%%%%%%%%%%%%%%%%%%%%%%
%  SECTION FORMULATION  %
%%%%%%%%%%%%%%%%%%%%%%%%%

\section{Formulation}

Here we present the basic formulation of a PG problem.

The free-flyer, the articular space and the forces being the usual variables.

Presentation of all the constraints and their formulation
\begin{itemize}
  \item Joint limits
  \item Auto-collision
  \item Collisions with environment
  \item Contacts with environment
  \item Stability
  \item Torque limits
\end{itemize}

