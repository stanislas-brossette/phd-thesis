
\chapter{State of the art and Problem definition}
\label{cha:state_of_the_art_and_problem_definition}

\section{State of the art}
\label{sec:state_of_the_art}

Posture Generalization can be viewed as, and is sometimes called, Generalized Inverse Kinematics.
The Inverse Kinematics problem consists in finding the joint configuration for an articulated multibody to complete a given task.
It is, by definition purely kinematics, it has no regards for stability, or other physics related constraints.
The IK problem has been widely studied and used in the fields of robotics, computer graphics, computer games and animation.
For the simplest cases, with robotic arms that have less than 7 degrees of freedom, a closed-form solution can be found.
But for more complicated cases, optimization methods are usually used.

The Generalized Inverse Kinematics refers to a problem similar to the Inverse Kinematics in the sense that it searches a joint configuration for an articulated figure to complete a task under several other constraints like ensuring the stability of the structure, respecting its joint limits, avoiding collision with the environment or with itself.

Generating desired initial, intermediary or finale posture configurations requires defining static task goals (e.g.\ reach a target point in 6D) to be done under intrinsic constraints such as joint limits, torque limits, avoiding non-desired self-collisions\ldots and perceptual or extrinsic ones such as keeping an object in the embedded camera field-of-view, avoiding non-desired collisions with surrounding objects, etc.

Most approaches to solve IK problems use pseudo-inverse based methods that are the Jacobian inverse and its variations, Jacobian transpose, Damped least squares with and without Singular Values decomposition or Selectively Damped Least Squares~\cite{balestrino1984robust, tolani2000real, baillieul1985kinematic, wampler1986manipulator, nakamura1986inverse, buss2005selectively}.
Those approaches are all computationally expensive and suffer from singularities.
In~\cite{pechev2008inverse} an alternative method based on a control approach is proposed.
Some statistical resolution approaches have been proposed in~\cite{courty2008inverse, hecker2008real}, with respectively a sequential Monte Carlo method and a particle filtering approach.
In~\cite{AristidouFABRIK, Aristidou:2016_ExtFABRIK} the forward and backward reaching inverse kinematics resolution method that is based on a geometric iterative heuristic approach is proposed.
Finding a solution to an IK problem without regards for the stability of the solution is a popular approach in the field of randomized path planning~\cite{cortes2002random, lavalle1999probabilistic}.

All these approaches are used to find a robot configuration $q$ solution to the geometric IK problem.
That solution can then be tested for stability, using methods such as the ones presented in~\cite{bretl:itro:2008} or~\cite{rimon2008general} that determine if the configuration $q$ can be in static equilibrium.
This gives a rejection criterion for the proposed solution.
Then if it is deemed feasible, the optimal contact forces can be computed using method proposed in~\cite{boyd2007fast}.
Those approach lead to a sequential resolution of the posture generation problem where a solution to the IK problem is searched, then tested for stability, if it is stable, the optimal forces are computed, otherwise, another solution is searched for.

In~\cite{Zhao1994}, Zhao proposes to use a nonlinear optimization algorithm over a Cartesian space to solve the IK problem.
\cite{bouyarmane2010static} extends that approach to solve the complete posture generation problem with a single nonlinear constrained optimization query, computing the contact forces and joint configuration at the same time, considering all the limitations of the posture generation problem as constraints in the optimization problem.


%\cite{aristidou2009} presents a review of existing techniques to solve inverse kinematics and one can see that the optimization approaches to that problem are various: Inverse Jacobian, Newton method, Sequential Monte Carlo and Heuristic approaches. It also presents a novel geometric iterative heuristic approach.

%The Generalized Inverse Kinematics refers to a problem similar to the Inverse Kinematics in the sense that it searches a joint configuration for an articulated figure to complete a task under several other constraints like ensuring the stability of the structure, respecting its joint limits, avoiding collision with the environment or with itself.

%Generating desired initial, intermediary or finale posture configurations requires defining static task goals (e.g.\ reach a target point in 6D) to be done under intrinsic constraints such as joint limits, torque limits, avoiding non-desired self-collisions\ldots and perceptual or extrinsic ones such as keeping an object in the embedded camera field-of-view, avoiding non-desired collisions with surrounding objects, etc.

In the past few years, our team has dedicated considerable efforts in proposing a general multi-contact motion planner to solve such cases of non-gaited acyclic planning.
Given a humanoid robot, an environment, a start and a final desired postures, the planner generates a sequence of contact stances allowing any part of the humanoid to make contact with any part of the environment to achieve motion towards the goal.
The planner's role is to grow a tree of contact stances iteratively, from a given posture, it tries to removes one of its contacts or to add a new one.
The tree grows, following some heuristics until the solution is reached.
A typical experiment with a HRP-2 robot achieving such an acyclic motion is presented in~\cite{escande:iser:2008}, and the planner is thoroughly described in~\cite{escande:ras:2013}.
Extensions of this multi-contact planner to multi-agent robots and objects gathering locomotion and manipulation are presented in~\cite{bouyarmane:ar:2012}, and preliminary validations with some DARPA challenge scenarios, such as climbing a ladder, ingress/egress a utility car or crossing through a relatively constrained pathway are presented in~\cite{bouyarmane:humanoids:2012}.
\cite{hauser:issr:2007} presents a different approach to multi-contact planning based on probabilistic roadmap and random sampling of the configuration space.
Another way of planning a multi-contact scenario, which is actually the most popular, is to do it by hand, the user chooses iteratively which contacts to add and remove until the goal is reached.

We dedicated considerable efforts in proposing a general multi-contact motion planner to solve cases of non-gaited acyclic planning.
Given a humanoid robot, an environment, a start and a final desired postures, our planner generates a sequence of contact stances allowing any part of the humanoid to make contact with any part of the environment to achieve motion towards the goal.
Our planner is thoroughly described in~\cite{escande:ras:2013}.
Extensions of this multi-contact planner to multi-agent robots and objects gathering locomotion and manipulation are presented in~\cite{bouyarmane:ar:2012}, and preliminary validations with some DARPA challenge scenarios, such as climbing a ladder, ingress/egress a utility car or crossing through a relatively constrained pathway are presented in~\cite{bouyarmane:humanoids:2012}.
In~\cite{escande:ras:2013} and~\cite{bouyarmane:ar:2012}, we describe works in multi-contact that are achieved by other colleagues in robotics.

\section{Problem Definition}
\label{sec:problem_definition}

%%%%%%%%%%%%%%%%%%%%%%%%%%%%%%%%%%%%%%%%%%%%%%%%%%%%%%%%%%%%%%%%%%%%%%%
%                     SECTION PROBLEM DEFINITION                      %
%%%%%%%%%%%%%%%%%%%%%%%%%%%%%%%%%%%%%%%%%%%%%%%%%%%%%%%%%%%%%%%%%%%%%%%

We consider the problem where we have a robotic system and we want to have it do some tasks, like for example to make a contact between a point on a body of the robot and a point of the environment.
This contact task can be modelized as a simple equality.
We denote $q$ the joint configuration (joint + base) of the robot, $g(q)$ is the 3D position of the point of interest on the robot and $P$ is the point of the environment.
Then our problem comes down to finding a configuration $q^*$ such that $f(q^*) = P$.
For robots like humanoids, the equations describing their kinematics are quite complicated, and, in the presence of angular joints, they are nonlinear.
In the general case, even for simple tasks, a closed form solution of the problem does not exist.

We denote $\mathcal{C}$ the configuration space of our robotic system.
It is the manifold in which $q$ lives. The dimension of $\mathcal{C}$ is equal to the number of degrees of freedom of the robot.
Note that if all the joints of the robot are actuated, this is the number of motors of the robot.
And if the robot has a free-floating basis (is not fixed) then the position in $\mathbb{R}^3$ and rotation in $SO(3)$ of its basis are part of $q$.

We can formulate that problem as follows:

\begin{equation}
  \text{find}\ q\in\mathcal{C}\ \text{such that}\ g(q)=P
\end{equation}

The space of solution of this problem is a submanifold of $\mathcal{C}$ that we call ${\mathcal{C}}_F$ the feasible configuration space (which can be empty).

In addition to the equality constraint abovementioned, the problem can feature some inequality constraints.
For example, each joint variable can be limited to a certain range of value: $\forall i,\ q_i\in [q_i^-,\ q_i^+]$. Then the problem becomes a combination of equality and inequality constraints.%, and can be written as:
%\begin{equation}
  %\text{find}\ q\in\mathcal{C}\ \text{such that}\left\{
  %\begin{array}{l}
    %g(q)=P \\
    %\forall i,\ q_i^- \leq q_i \leq q_i^+
  %\end{array}
  %\right.
%\end{equation}
Finally, we can also require finding the point of $\mathcal{C}_F$ that minimizes a criterion, such as the distance to a goal posture $q_0$.
Our problem becomes:

\begin{equation}
  \left\{
  \begin{array}{l}
    \minimize\limits_{q\in\mathcal{C}}{\|q-q_0\|_2}\\
    \text{ s.t. }
    \left\{
    \begin{array}{l}
      g(q)=P\\
      \forall i,\ q_i^- \leq q_i \leq q_i^+
    \end{array}
    \right.
  \end{array}
  \right.
\end{equation}

This type of problem is called a nonlinear constrained optimization problem and can be formulated in a more generic fashion as:

\begin{equation}
\label{eq:optim_form_PG}
  \left\{
  \begin{array}{l}
    \minimize\limits_{x\in\mathcal{C}}{f(x)}\\
    \text{ s.t. }
    \left\{
    \begin{array}{l}
      c_i(x) = 0,\ \forall i\in{E}\\
      c_i(x) \geq 0,\ \forall i\in{I}\\
    \end{array}
    \right.
  \end{array}
  \right.
\end{equation}

Where $x$ is the optimization variable (in space $\mathcal{C}$) we want to find that minimizes the cost function $f(x)$ while satisfying the equality constraints $c_i(x) = 0,\ \forall i\in{E}$ and inequality constraints $c_i(x) \geq 0,\ \forall i\in{I}$

Such problems can be solved by a nonlinear optimization solver.
In Appendix~\ref{chapter:optimization}, we present some principles of nonlinear optimization in unconstrained and constrained cases.
Several off-the-shelf solvers are available and have been widely used in for solving robotics problems.
The CFSQP solver~\cite{cfsqp:manual} was used in~\cite{escande:iros:2009} and~\cite{escande:ras:2013} where thousands of HRP-2 posture generation queries were made to explore the feasible space.
The IPOPT solver~\cite{wachter:mathprog:2006} has been used in~\cite{vaillant:humanoids:2014},~\cite{vaillant:autonomousrobots:2016},~\cite{bouyarmane:ar:2012} where the posture generator had been extended to handle multi-robot problems and more complex and various contact models.
The SNOPT solver~\cite{gill2005snopt} was used in~\cite{dai2014whole} to plan dynamic whole-body trajectories.
Several nonlinear optimization solvers are available and have been packaged for use in robotics problems in the Roboptim Framework~\cite{moulard:jsme:2013}, such as IPOPT, CFSQP, CMinPack~\cite{cminpack}, NAG~\cite{nag}, KNITRO~\cite{knitro} and PGSolver, the solver that we develop in this thesis.

In this thesis, we use the off-the-shelf solvers IPOPT~\cite{wachter:mathprog:2006} in the beginning and later we tackle the development of our own nonlinear optimization solver and then use it.

From this point forward, we formulate and solve posture generation problems as nonlinear constrained optimization problem.
And in the next chapter, we focus on formulating several robotics constraints and cost function in the formalism of nonlinear optimization.
