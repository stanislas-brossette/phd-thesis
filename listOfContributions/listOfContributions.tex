\documentclass{article}
\usepackage{amsfonts}
\usepackage{amsmath}
\usepackage{algorithm}
\usepackage{algpseudocode}
\usepackage{hyperref}
\usepackage{graphicx}
\graphicspath{ {images/} }
\begin{document}

\title{List of contributions}
\author{Stanislas Brossette}

\maketitle

\paragraph {Contact Planning in Real environment}
\begin{itemize}
  \item Understanding the environment through a point-cloud
    Segmentation pipeline:
    \begin{itemize}
      \item Acquisition
      \item Filtering
      \item Region growing segmentation
      \item Planar Extraction
      \item Planar projection and Hull convex generation
      \item Re-orientation and transfer to the planner
    \end{itemize}

  \item Contact generation with convex surface inclusions
Usual methods for generating surface contact are based on point-to-point sampling, on rectangular inclusion, or other limiting methods. We proposed to extend that to the inclusion of convex surfaces.
  \item{Simulated scenarios planned on real-environment}
We present 2 planned scenarios with the HRP-2 robot
\end{itemize}



%%%%%%%%%%%%%%%%%%%%%%%%%%%%%%%%%%%%%%%%%%%%%%%%%%%%%%%%%%%%%%%%%%%%%%%%%
%  SECTION INTEGRATION ON NON-INCLUSIVE CONTACTS IN POSTURE GENERATION  %
%%%%%%%%%%%%%%%%%%%%%%%%%%%%%%%%%%%%%%%%%%%%%%%%%%%%%%%%%%%%%%%%%%%%%%%%%

\paragraph{Integration on Non-Inclusive Contacts in Posture Generation}

The method presented before is still limiting in many cases. Ladder, stairs climbing\dots.
Sometimes it is not possible to ensure inclusion of 2 surfaces.

\begin{itemize}
  \item Contact geometry formulation
  \item Present the difficulties related to intersecting 2 convex polygons. Non-constant number of constraints.

  \item{Non inclusive contact constraints}
    \begin{itemize}
      \item {Main Idea}
      \item {Pseudo-distance}
      \item {Modification of the optimization problem}
      \item {Maximixation of the contact area}
      \item {Using non inclusuve contact to maintain stability}
      \item {Extension to singular cases}
    \end{itemize}
  \item{Simulation Results}
     \begin{itemize}
      \item{Inclined ladder climbing}
      \item{Vertical ladder climbing}
      \item{Climbing Stairs}
      \item{Walking along a path made of small objects}
     \end{itemize}
\end{itemize}

%%%%%%%%%%%%%%%%%%%%%%%%%%%%%%%%%%%%%%%%%%%%%%%%%%%%%%%%%%%%%%%%%%%%
%  ON THE USE OF LIFTED VARIABLES FOR ROBOTICS POSTURE GENERATION  %
%%%%%%%%%%%%%%%%%%%%%%%%%%%%%%%%%%%%%%%%%%%%%%%%%%%%%%%%%%%%%%%%%%%%

\paragraph{On the use of lifted variables for Robotics Posture Generation}

\begin{itemize}
  \item {Introduction}
  \item {Lifting Algorithm}
  \item {Optimization on lifted variables}
  \item {Condensed BFGS update}
  \item {Results, experimentation}
  \item {conclusion}
\end{itemize}
\end{document}
