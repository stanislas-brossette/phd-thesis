\documentclass{article}
%\usepackage{amsfonts}
%\usepackage{amsmath}
%\usepackage{algorithm}
%\usepackage{algpseudocode}
%\usepackage{hyperref}
%\usepackage{graphicx}
%\graphicspath{ {images/} }
\begin{document}

\title{List of contributions}
\author{Stanislas Brossette}

\maketitle


%%%%%%%%%%%%%%%
%  CHAPTER 1  %
%%%%%%%%%%%%%%%

\paragraph {Chapter 1: Posture generation, state of the art/Introduction}
\begin{itemize}
  \item Generalities, introduction
  \item Presentation of the existing methods
  \item From Inverse Kinematics to Generalized IK/posture Generation/pose estimation (addition of articular limits, forces, stability etc.
  \item Topology of the parametrization space (Free-flyer, q, f, other)
  \item Formulation as a non-linear constrained optimization problem
  \item Adrien \& Karim's formulations
  \item Formulation of several types of cost/constraints
  \begin{itemize}
    \item Contact with plane surface
    \item Collision avoidance
    \item Auto-Collision avoidance
    \item Static equilibrium: Newton/CoM projection
    \item Forces in friction cones
    \item Articular limits
    \item Torque limits
    \item Torque minimization
    \item Goal Posture
  \end{itemize}
  \item Reasons why it is not enough and why we needed a new PG
    \begin{itemize}
      \item Having an easier way to formulate problems
      \item Avoid having to de some gymnastic to remain on manifolds
      \item Automatic variable management
      \item Robustness
    \end{itemize}
  \item Utilization of posture generation in planning
\end{itemize}


%%%%%%%%%%%%%%%
%  CHAPTER 2  %
%%%%%%%%%%%%%%%

\paragraph {Chapter 2: Numerical optimization: Introduction}
\begin{itemize}
  \item Introduction to optimization
  \item Notions of unconstrained optimization
  \item Notions of Constrained Optimization
  \item KKT
  \item Line Search
  \item Trust Region
  \item Filter
  \item Merit function
  \item SQP
\end{itemize}


%%%%%%%%%%%%%%%
%  CHAPTER 3  %
%%%%%%%%%%%%%%%

\paragraph {Chapter 3: Some extensions of posture generation}
\subparagraph{Integration on Non-Inclusive Contacts in Posture Generation}

The method presented before is limiting in many cases. Ladder, stairs climbing\dots.
Sometimes it is not possible to ensure inclusion of 2 surfaces.

\begin{itemize}
  \item Contact geometry formulation
    \begin{itemize}
      \item Discretisation of contact surfaces for practical reasons
      \item Contact generation with convex surface inclusions
Usual methods for generating surface contact are based on point-to-point sampling, on rectangular inclusion, or other limiting methods. We extend that to the inclusion of convex surfaces.
      \item Inclusive contacts are no problem
      \item Non-Inclusive contacts lead to non-constant number of constraints or non-smooth gradients, cannot be solved with usual solvers with that formulation
    \end{itemize}
  \item{Non inclusive contact constraints}
    \begin{itemize}
      \item {Main Idea: Inserting an ellipse in the intersection of polygons}
      \item {Pseudo-distance(to simplify the formulation)}
      \item {Modification of the optimization problem}
      \item {Maximization of the contact area}
      \item {Using non inclusive contact to maintain stability}
      \item {Extension to singular cases}
    \end{itemize}
  \item{Simulation Results}
    \begin{itemize}
      \item{Inclined ladder climbing}
      \item{Vertical ladder climbing}
      \item{Climbing Stairs}
      \item{Walking along a path made of small objects}
    \end{itemize}
  \item{Application to ladder climbing (cf. Joris papers)}
\end{itemize}

\subparagraph{On the use of lifted variables for Robotics Posture Generation}
\begin{itemize}
  \item {Introduction: "J. Albersmeyer and M. Diehl, The lifted Newton Method and its Application in Optimisation"}
  \item {Lifting Algorithm: Introduce additional variables to reduce the complexity/degree of the equations to solve. And then use a trick to re-condense the system and avoid loosing computation time}
  \item {Optimization on lifted variables}
  \item {Condensed BFGS update}
  \item {Results, experimentation}
  \begin{itemize}
    \item Symbolic Inverse Kinematics
    \item Automatic Lifting algorithm adapted to posture generation
    \item Several Solvers for lifted and non-lifted systems (Gauss Newton, SQP, lifted Newton.... )
    \item Compared many optimization methods (Globalization, regularization....)
    \item Never managed to get results faster than basic methods
  \end{itemize}
  \item{Motivation:Development of a posture generator with a specialized solver}
    \begin{itemize}
      \item Problem
        \begin{itemize}
          \item Robotics problems solved with generic solvers (IPOPT, CFSQP, fmincon...)
          \item Robotics problems(especially posture generation) are very specific (Constraints, costs, working spaces, sizes...)
        \end{itemize}
      \item Solution
        \begin{itemize}
          \item Develop a posture generator, along with a non-linear solver
          \item Better understanding of our own solver
          \item Ability to use the solver in new ways:
          \begin{itemize}
            \item Working on Manifolds
            \item Variable numbers of constraints
            \item Automatic management of variables, constraints and mathematical expressions
          \end{itemize}
        \end{itemize}
    \end{itemize}
\end{itemize}


%%%%%%%%%%%%%%%
%  CHAPTER 4  %
%%%%%%%%%%%%%%%

\paragraph {Chapter 4: Optimisation on non-Eclidean manifolds }
\begin{itemize}
  \item{Optimisation on Manifolds}
    \begin{itemize}
      \item Definition and examples of non-Euclidean manifolds
        \begin{itemize}
          \item Rn
          \item SO(3)
          \item S2
          \item Cartesian Products
        \end{itemize}
      \item Representation space vs Tangent space vs manifold
      \item Local parametrization, notion of retractation
      \item Computation of function's derivatives, differentiation of retractation
      \item Hessian update on manifolds, notion of vector transport and application to BFGS and SR1 updates
    \end{itemize}
  \item{Practical Implementation: PGSolver}
    \begin{itemize}
      \item General SQP algorithm adapted to local parametrization of manifolds.
      \item Local QP and resolution with LSSOL
      \item Feasibility problems and resolution with LSSOL
      \item Acceptance criterion:
        \begin{itemize}
          \item Merit function
          \item Filter
        \end{itemize}
      \item Globalization method:
        \begin{itemize}
          \item Line-Search
          \item Trust-Region
            \begin{itemize}
              \item Notions of limit-Map vs trust-region vs problem's boundaries
              \item Decisions of increase or decrease of trust region's size
              \item Anisotrope trust-region
            \end{itemize}
        \end{itemize}
      \item Hessian Update:
        \begin{itemize}
          \item BFGS
          \item SR1
          \item Fletcher LQNU
          \item Individual vs grouped updates
          \item Limited memory updates
        \end{itemize}
      \item Termination conditions:
        \begin{itemize}
          \item Satisfaction of KKT constraints
          \item Null step
          \item Failure of QP, FP, other\dots
        \end{itemize}
      \item Restoration Phase:
        \begin{itemize}
          \item Problem to solve: minimization of violation, relaxation of violated cstr
          \item Exit condition: Feasible problem
          \item Second-order correction phase
        \end{itemize}
    \end{itemize}
  \item{Applications and testings of PGSolver on non-robotic problems}
    \begin{itemize}
      \item Point cloud fitting
      \item Surface parametrisation
      \item Cube stacks
      \item Schitkowsky
    \end{itemize}
\end{itemize}


%%%%%%%%%%%%%%%
%  CHAPTER 5  %
%%%%%%%%%%%%%%%

\paragraph{Chapter 5: New Posture Generation}
\begin{itemize}
  \item{Posture Generation, variables and architecture}
    \begin{itemize}
      \item Geometric expressions:
        \begin{itemize}
          \item Represent everything based on frames
          \item Automatic derivative computation
        \end{itemize}
      \item Automatic mapping
      \item Problem generator: a posture generation problem factory
    \end{itemize}
  \item{Problem Formulation: formulation of usual problems in the geometric expressions framework}
    \begin{itemize}
      \item Contact with plane surface
      \item Static equilibrium: Newton/CoM projection
      \item Forces in friction cones
      \item Articular limits
      \item Torque limits
      \item Torque minimization
      \item Goal Posture
    \end{itemize}
  \item{Implementation of new types of constraints in our formulation}
    \begin{itemize}
      \item Contact with parametrized surfaces on Rn
      \item Contact with parametrized surfaces on S2: Sphere, SuperEllipsoid
      \item Contact with Catmull-Clark Subdivision surfaces
    \end{itemize}
  \item{Simulation Results}
\end{itemize}

%%%%%%%%%%%%%%%
%  CHAPTER 6  %
%%%%%%%%%%%%%%%

\paragraph{Chapter 6: Forces in posture generation}
\begin{itemize}
  \item{Contact with desired force}
  \item{Potential contacts}
\end{itemize}


%%%%%%%%%%%%%%%
%  CHAPTER 7  %
%%%%%%%%%%%%%%%

\paragraph{Chapter 7: Evaluation of the PG, experimentation}
\begin{itemize}
  \item Understanding the environment through a point-cloud
    Extraction of plane polygons from Kinect acquired point-clouds.

    Segmentation pipeline:
    \begin{itemize}
      \item Acquisition
      \item Filtering
      \item Region growing segmentation
      \item Planar Extraction
      \item Planar projection and Hull convex generation
      \item Re-orientation and transfer to the planner
    \end{itemize}
  \item Generation of sequences of postures on segmented environment
  \item{Simulated scenarios planned on real-environment: HRP-2 climbing on table through stairs, HRP-2 climbing on table through slope}
  \item{Optimization of the solvers parameter for a class of problems}
    \begin{itemize}
      \item Developpement of a list of typical robotics optimization problem
      \item Training of a genetic algorithm to find the "best" solver tuning to solve our list of problems
    \end{itemize}
  \item{Several scenarios, including Airbus}
\end{itemize}


\paragraph{Optional: Inertial Identification: Minimal parametrization of Inertia matrix that ensures physical consistency}


\end{document}
