\documentclass{article}
\usepackage{amsfonts}
\usepackage{amsmath}
\usepackage{algorithm}
\usepackage{algpseudocode}
\usepackage{hyperref}
\usepackage{graphicx}
\graphicspath{ {images/} }
\begin{document}

\title{List of contributions}
\author{Stanislas Brossette}

\maketitle

\paragraph {Posture generation: state of the art/Introduction}
\begin{itemize}
  \item Generalities, introduction
  \item Topology of the parametrization space (Free-flyer, q, f, other)
  \item Formulation as a non-linear constrained optimization problem
  \item Formulation of several types of cost/constraints
  \begin{itemize}
    \item Contact with plane surface
    \item Collision avoidance
    \item Auto-Collision avoidance
    \item Static equilibrium: Newton/CoM projection
    \item Forces in friction cones
    \item Articular limits
    \item Torque limits
    \item Torque minimization
    \item Goal Posture
  \end{itemize}
  \item  Utilization of posture generation in planning
\end{itemize}

\paragraph {Numerical optimization: State of the art/Introduction}
\begin{itemize}
  \item Generalities, introduction
\end{itemize}

\paragraph {Contact Planning in Real environment}
\begin{itemize}
  \item Understanding the environment through a point-cloud
    Segmentation pipeline:
    \begin{itemize}
      \item Acquisition
      \item Filtering
      \item Region growing segmentation
      \item Planar Extraction
      \item Planar projection and Hull convex generation
      \item Re-orientation and transfer to the planner
    \end{itemize}
  \item Contact generation with convex surface inclusions
Usual methods for generating surface contact are based on point-to-point sampling, on rectangular inclusion, or other limiting methods. We extend that to the inclusion of convex surfaces.
  \item{Simulated scenarios planned on real-environment: HRP-2 climbing on table through stairs, HRP-2 climbing on table through slope}
\end{itemize}


%%%%%%%%%%%%%%%%%%%%%%%%%%%%%%%%%%%%%%%%%%%%%%%%%%%%%%%%%%%%%%%%%%%%%%%%%
%  SECTION INTEGRATION ON NON-INCLUSIVE CONTACTS IN POSTURE GENERATION  %
%%%%%%%%%%%%%%%%%%%%%%%%%%%%%%%%%%%%%%%%%%%%%%%%%%%%%%%%%%%%%%%%%%%%%%%%%

\paragraph{Integration on Non-Inclusive Contacts in Posture Generation}

The method presented before is limiting in many cases. Ladder, stairs climbing\dots.
Sometimes it is not possible to ensure inclusion of 2 surfaces.

\begin{itemize}
  \item Contact geometry formulation
    \begin{itemize}
      \item Discretisation of contact surfaces for practical reasons
      \item Inclusize contacts are no problem
      \item Non-Inclusive contacts lead to non-constant number of constraints, cannot be solved with that formulation
    \end{itemize}
  \item{Non inclusive contact constraints}
    \begin{itemize}
      \item {Main Idea: Inserting an ellipse in the intersection of polygons}
      \item {Pseudo-distance}
      \item {Modification of the optimization problem}
      \item {Maximixation of the contact area}
      \item {Using non inclusive contact to maintain stability}
      \item {Extension to singular cases}
    \end{itemize}
  \item{Simulation Results}
    \begin{itemize}
      \item{Inclined ladder climbing}
      \item{Vertical ladder climbing}
      \item{Climbing Stairs}
      \item{Walking along a path made of small objects}
    \end{itemize}
\end{itemize}

%%%%%%%%%%%%%%%%%%%%%%%%%%%%%%%%%%%%%%%%%%%%%%%%%%%%%%%%%%%%%%%%%%%%
%  ON THE USE OF LIFTED VARIABLES FOR ROBOTICS POSTURE GENERATION  %
%%%%%%%%%%%%%%%%%%%%%%%%%%%%%%%%%%%%%%%%%%%%%%%%%%%%%%%%%%%%%%%%%%%%

\paragraph{On the use of lifted variables for Robotics Posture Generation}

\begin{itemize}
  \item {Introduction}
  \item {Lifting Algorithm}
  \item {Optimization on lifted variables}
  \item {Condensed BFGS update}
  \item {Results, experimentation}
  \item {conclusion}
\end{itemize}

\paragraph { Posture generation and Optimisation on non-Eclidean manifolds }
\begin{itemize}
  \item{Introduction}
  \item{Optimisation on Manifolds Theory}
  \item{Practical Implementation: PGSolver}
  \item{Posture Generation, variables and architecture}
  \item{Problem Formulation}
  \item{Simulation Results}
  \item{Parametrization of complex solids on S2}
  \item{Potential contacts}
  \item{Optimization of the solvers parameter for a class of problems}
\end{itemize}




\end{document}
